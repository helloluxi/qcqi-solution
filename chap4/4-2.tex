\chapter{Quantum circuits}

\section{Quantum algorithms}

\section{Single qubit operations}

\ex $X:(\pm 1,0,0)$, $Y:(0,\pm 1,0)$, $Z:(0,0,\pm 1)$.

\ex Follow (Ex 2.35).

\ex $T=e^{i\pi/8}R_z(\pi/4)$.

\ex $H=iR_x(\frac{\pi}{2})R_z(\frac{\pi}{2})R_x(\frac{\pi}{2})$, by imaging a process to swap $\ket{0}$ and $\ket{+}$ on the Bloch sphere.

\ex Follow (Ex 2.35).

\ex Let $\ket{\lambda}$ be the state whose Bloch vector is $\vec{\lambda}$, and let $U$ be an unitary transform such that $U\ket{0} = \ket{\lambda}$, where $\ket{0}$ is the state whose Bloch vector is $\hat{z}$.
Then 
$$U\frac{I+\hat{z}\cdot\vec{\sigma}}{2}U^\dagger = U\dyadic{0}{0}U^\dagger = \dyadic{\lambda}{\lambda} = \frac{I+\vec{\lambda}\cdot\vec{\sigma}}{2}.$$

Therefore,
$$U(\hat{z}\cdot\vec{\sigma})U^\dagger = \vec{\lambda}\cdot\vec{\sigma}.$$

Hence
$$R_n(\theta) = e^{-\frac{i\theta}{2}\vec{n}\cdot\vec{\sigma}} = e^{-\frac{i\theta}{2}U(\hat{z}\cdot\vec{\sigma})U^\dagger} = Ue^{-\frac{i\theta}{2}\hat{z}\cdot\vec{\sigma}} = UR_z(\theta)U^\dagger.$$

So we can interpret $R_n(\theta)$ as: first rotate the Bloch sphere such that the vector $\vec{\lambda}$ arrives $\hat{z}$, then rotate the Bloch sphere around $\hat{z}$ by angle $\theta$, finally rotate it such that $\hat{z}$ comes back to $\vec{\lambda}$, which is clearly identical to rotate around $\vec{n}$ by angle $\theta$.

\ex From $\sigma_a\sigma_b=i\varepsilon_{abc}\sigma_c(a,b,c\in\{1,2,3\})$ we know $XYX=iZX=i(iY)=-Y$, hence
$$\begin{aligned}
    XR_y(\theta)X=&X\left[\sum_n\frac{1}{n!}\left(-\frac{i\theta Y}{2}\right)^n\right]X
    \\=&\sum_n\frac{1}{n!}\left(-\frac{i\theta}{2}\right)^nXY^nX
    \\=&\sum_n\frac{1}{n!}\left(-\frac{i\theta}{2}\right)^n(XYX)^n
    \\=&\sum_n\frac{1}{n!}\left(\frac{i\theta Y}{2}\right)^n
    \\=&R_y(-\theta)\end{aligned}$$.

\ex (1) For arbitrary $2\times 2$ unitary matrix $U$, we know earlier that $\det(U)=\exp(i2\alpha)$ for some $\alpha\in\mathbb{R}$. Then $U/\exp(i\alpha)$ is a special unitary matrix, which we now prove to be equal to some $R_n(\theta)$. Since unitary matrices are diagonalizable, we have
$$\begin{aligned}
    U/\exp(i\alpha) = & V^\dagger \begin{pmatrix}
        e^{i\theta/2} & 0 \\
        0 & e^{-i\theta/2}
    \end{pmatrix}V
    \\ = & V^\dagger R_z(\theta) V
    \\ = & V^\dagger \exp(-i\theta Z/2) V
    \\ = & \exp(-i\theta V^\dagger ZV/2)
\end{aligned}$$
for some special unitary matrix $V$.
Since $V^\dagger ZV$ is Hermitian and has trace zero, it can be linearly represented by $\{X,Y,Z\}$, as $n_xX+n_yY+n_zZ$. Denote $\vec{n}=(n_x,n_y,n_z)$, we finally have
$$
    U/\exp(i\alpha) = \exp(-i\theta (n_xX+n_yY+n_zZ)/2) = R_n(\theta)
$$

(2) $\alpha=-\frac{\pi}{2}$, $\theta=\pi$, $\vec{n}=\frac{1}{\sqrt{2}}(1,0,1)$.

(3) $\alpha=\frac{\pi}{4}$, $\theta=\frac{\pi}{2}$, $\vec{n}=\hat{z}$.

\ex Since $U$ is unitary, the rows and columns of $U$ are normalized, so we can write $U$ as
$$
    U = \begin{pmatrix}
        e^{ia}\cos\frac{\theta}{2} & e^{ic}\sin\frac{\theta}{2} \\
        e^{ib}\sin\frac{\theta}{2} & e^{id}\cos\frac{\theta}{2}
    \end{pmatrix}, \qquad(a,b,c,d\in\mathbb{R}).
$$
By orthogonal property we have $(e^{i(a-c)}-e^{i(b-d)})\cos\frac{\theta}{2}\sin\frac{\theta}{2} = 0$.

In general position where $\cos\frac{\theta}{2}\sin\frac{\theta}{2} \neq 0$, we add a constraint $a-c=b-d$. By comparing we have
$$\alpha = \frac{1}{4}(a+b+c+d),\quad
\beta = \frac{1}{2}(-a-b+c+d),\quad
\gamma = \theta,\quad
\delta = \frac{1}{2}(-a+b-c+d).$$

For special situation it is even earlier and we skip.

\ex $$U = e^{i\alpha}R_z(\beta)R_x(\gamma)R_z(\delta) = \begin{pmatrix}
    e^{i\left(\alpha-\frac{\beta}{2}-\frac{\delta}{2}\right)}\cos\frac{\gamma}{2} & -ie^{i\left(\alpha-\frac{\beta}{2}+\frac{\delta}{2}\right)}\sin\frac{\gamma}{2} \\
    ie^{i\left(\alpha+\frac{\beta}{2}-\frac{\delta}{2}\right)}\sin\frac{\gamma}{2} & e^{i\left(\alpha+\frac{\beta}{2}+\frac{\delta}{2}\right)}\cos\frac{\gamma}{2}
\end{pmatrix}.$$

\ex \redstar  Suppose $\vec{m}$ and $\vec{n}$ are very close, choose a unit vector that is close to both of them, then it is impossible to transform it to anywhere on the unit sphere within three rotations?

\ex Notice that the Hadamard gate acts on a Bloch sphere by swapping $\ket{0}$ and $\ket{+}$, we know an Hadamard transform is equal to first rotate around $\hat{z}$ by $\pi$, then rotate around $\hat{y}$ by $\pi/2$.
Thus $H=iR_z(0)R_y(\frac{\pi}{2})R_z(\pi)$.
And we can write directly that
$$\alpha=\pi,\quad A=R_y(\frac{\pi}{4}),\quad B=R_y(-\frac{\pi}{4})R_z(-\frac{\pi}{2}),\quad C=R_z(\frac{\pi}{2})$$

\ex pass. (It is useful that $H=(X+Z)/\sqrt{2}$.)

\ex $$\begin{aligned}HTH=&e^{i\frac{\pi}{8}}HR_z(\frac{\pi}{4})H
    \\=&e^{i\frac{\pi}{8}}\sum_n\frac{1}{n!}H(-i\pi Z/8)^nH
    \\=&e^{i\frac{\pi}{8}}\sum_n\frac{1}{n!}(-i\pi HZH/8)^n
    \\=&e^{i\frac{\pi}{8}}\sum_n\frac{1}{n!}H(-i\pi X/8)^nH
    \\=&e^{i\frac{\pi}{8}}R_x(\frac{\pi}{4})\end{aligned}$$.

\ex pass.

