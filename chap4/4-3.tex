\section{Controlled operations}

\ex $$\frac{1}{\sqrt{2}}\begin{pmatrix}
    1 & 1 & & \\
    1 & -1 & & \\
    & & 1 & 1 \\
    & & 1 & -1
\end{pmatrix},\qquad\frac{1}{\sqrt{2}}\begin{pmatrix}
    1 & & 1 & \\
    & 1 & & 1 \\
    1 & & -1 & \\
    & 1 & & -1
\end{pmatrix}$$.

\ex Using $HIH=I$ and $HZH=X$ we have

$$\Qcircuit @C=1em @R=.7em {
    & \qw & \ctrl{1} & \qw & \qw \\
    & \gate{H} & \gate{Z} & \gate{H} & \qw
}$$

\ex They both obtain the following transformation.
$$\ket{00}\rightarrow\ket{00},\qquad\ket{01}\rightarrow\ket{01},\qquad\ket{10}\rightarrow\ket{10},\qquad\ket{11}\rightarrow-\ket{11}$$

\ex $$\begin{pmatrix}
    \rho_{11} & \rho_{12} & \rho_{13} & \rho_{14} \\
    \rho_{21} & \rho_{22} & \rho_{23} & \rho_{24} \\
    \rho_{31} & \rho_{32} & \rho_{33} & \rho_{34} \\
    \rho_{41} & \rho_{42} & \rho_{43} & \rho_{44}
\end{pmatrix}\Rightarrow\begin{pmatrix}
    \rho_{11} & \rho_{12} & \rho_{14} & \rho_{13} \\
    \rho_{21} & \rho_{22} & \rho_{24} & \rho_{23} \\
    \rho_{41} & \rho_{42} & \rho_{44} & \rho_{43} \\
    \rho_{31} & \rho_{32} & \rho_{34} & \rho_{33}
\end{pmatrix}$$.

\ex pass.

\ex $$\begin{aligned}
    & \begin{pmatrix}
        I &   &   &   \\
        & V &   &   \\
        &   & I &   \\
        &   &   & V
    \end{pmatrix}\begin{pmatrix}
        I &   &   &   \\
        & I &   &   \\
        &   &   & I \\
        &   & I &  
    \end{pmatrix}\begin{pmatrix}
        I &           &   &   \\
        & V^\dagger &   &   \\
        &           & I &   \\
        &           &   & V^\dagger
    \end{pmatrix}\begin{pmatrix}
        I &   &   &   \\
        & I &   &   \\
        &   &   & I \\
        &   & I &  
    \end{pmatrix}\begin{pmatrix}
        I &   &   &   \\
        & I &   &   \\
        &   & V &   \\
        &   &   & V
    \end{pmatrix}
    \\ = & \begin{pmatrix}
        I &            &           &   \\
        & VV^\dagger &           &   \\
        &            & V^\dagger &   \\
        &            &           & V^2
    \end{pmatrix} = \begin{pmatrix}
        I &   &   &   \\
        & I &   &   \\
        &   & I &   \\
        &   &   & U
    \end{pmatrix}.
\end{aligned}$$

\ex Let $V^2=U$ and $V=AXBXC$, combining (Figure 4.6) and (Figure 4.8) in the book we get a circuit like

$$\Qcircuit @C=.7em @R=.5em {
    & \qw & \qw & \qw & \qw & \qw & \ctrl{1} & \qw & \qw & \qw & \qw & \qw & \ctrl{1} & \gate{\alpha} & \ctrl{2} & \qw & \ctrl{2} & \qw & \qw \\
    & \qw & \ctrl{1} & \qw & \ctrl{1} & \gate{\alpha} & \targ & \gate{-\alpha} & \ctrl{1} & \qw & \ctrl{1} & \qw & \targ & \qw & \qw & \qw & \qw & \qw & \qw \\
    & \gate{C} & \targ & \gate{B} & \targ & \gate{A} & \qw & \gate{A^\dagger} & \targ & \gate{B^\dagger} & \targ & \gate{C^\dagger} & \qw & \gate{C} & \targ & \gate{B} & \targ & \gate{A} & \qw 
    \gategroup{2}{2}{3}{6}{.5em}{--}
    \gategroup{2}{8}{3}{12}{.5em}{--}
    \gategroup{1}{14}{3}{18}{.5em}{--}
}$$
where $\alpha$-gate stands for the one-qubit gate $diag(1,e^{i\alpha})$.

By collapsing $AA^\dagger$ and $C^\dagger C$ on the third qubit wire and moving the $\alpha$-gate on the first qubit wire we get

$$\Qcircuit @C=.7em @R=.5em {
    & \qw & \qw & \qw & \qw & \gate{\alpha} & \ctrl{1} & \qw & \qw & \qw & \qw & \ctrl{1} & \ctrl{2} & \qw & \ctrl{2} & \qw & \qw \\
    & \qw & \ctrl{1} & \qw & \ctrl{1} & \gate{\alpha} & \targ & \gate{-\alpha} & \ctrl{1} & \qw & \ctrl{1} & \targ & \qw & \qw & \qw & \qw & \qw \\
    & \gate{C} & \targ & \gate{B} & \targ & \qw & \qw & \qw & \targ & \gate{B^\dagger} & \targ & \qw & \targ & \gate{B} & \targ & \gate{A} & \qw 
}$$

Notice that both the following circuits transfer $(x,y,z)$ into $(x,x\oplus y,x\oplus y\oplus z)$.

$$\Qcircuit @C=1em @R=.7em {
    & \qw & \ctrl{1} & \ctrl{2} & \qw \\
    & \ctrl{1} & \targ & \qw & \qw \\
    & \targ & \qw & \targ & \qw
} = \Qcircuit @C=1em @R=.7em {
    & \ctrl{1} & \qw & \qw \\
    & \targ & \ctrl{1} & \qw \\
    & \qw & \targ & \qw
}$$

Using this identity we can turn
$$\Qcircuit @C=.7em @R=.5em {
    & \qw & \qw & \qw & \qw & \gate{\alpha} & \ctrl{1} & \qw & \ctrl{1} & \ctrl{1} & \qw & \qw & \qw & \ctrl{1} & \ctrl{2} & \qw & \ctrl{2} & \qw & \qw \\
    & \qw & \ctrl{1} & \qw & \ctrl{1} & \gate{\alpha} & \targ & \gate{-\alpha} & \targ & \targ & \ctrl{1} & \qw & \ctrl{1} & \targ & \qw & \qw & \qw & \qw & \qw \\
    & \gate{C} & \targ & \gate{B} & \targ & \qw & \qw & \qw & \qw & \qw & \targ & \gate{B^\dagger} & \targ & \qw & \targ & \gate{B} & \targ & \gate{A} & \qw 
    \gategroup{1}{10}{3}{11}{.5em}{--}
    \gategroup{1}{13}{3}{15}{.5em}{--}
}$$
into
$$\Qcircuit @C=.7em @R=.5em {
    & \qw & \qw & \qw & \qw & \gate{\alpha} & \ctrl{1} & \qw & \ctrl{1} & \qw & \ctrl{2} & \ctrl{1} & \qw & \ctrl{1} & \qw & \qw & \ctrl{2} & \qw & \qw \\
    & \qw & \ctrl{1} & \qw & \ctrl{1} & \gate{\alpha} & \targ & \gate{-\alpha} & \targ & \ctrl{1} & \qw & \targ & \qw & \targ & \ctrl{1} & \qw & \qw & \qw & \qw \\
    & \gate{C} & \targ & \gate{B} & \targ & \qw & \qw & \qw & \qw & \targ & \targ & \qw & \gate{B^\dagger} & \qw & \targ & \gate{B} & \targ & \gate{A} & \qw 
    \gategroup{1}{6}{2}{9}{.5em}{--}
    \gategroup{1}{10}{3}{12}{.5em}{--}
    \gategroup{1}{14}{3}{15}{.5em}{--}
}$$

First we collapse the pair of CNOT gates above the $B^\dagger$-gate.
Since the gates in the first dashed frame implements the unitary operator $diag(1,1,1,e^{i2\alpha})$, we can move them to the very right of the circuit.
Finally another pair of CNOT gates come together and we collapse them, and we get

$$\Qcircuit @C=1em @R=.7em {
    & \qw & \qw & \qw & \ctrl{2} & \qw & \qw & \qw & \ctrl{2} & \qw & \gate{\alpha} & \ctrl{1} & \qw & \ctrl{1} & \qw \\
    & \qw & \ctrl{1} & \qw & \qw & \qw & \ctrl{1} & \qw & \qw & \qw & \gate{\alpha} & \targ & \gate{-\alpha} & \targ & \qw \\
    & \gate{C} & \targ & \gate{B} & \targ & \gate{B^\dagger} & \targ & \gate{B} & \targ & \gate{A} & \qw & & & & {} \gategroup{1}{11}{2}{14}{.7em}{--}
}$$

Reference: \url{https://quantumcomputing.stackexchange.com/questions/7082}

\ex (1) $U=R_x(\theta)$:

$$\Qcircuit @C=1em @R=.7em {
    & \qw & \ctrl{1} & \qw & \ctrl{1} & \qw & \qw \\
    & \gate{R_z(\frac{\pi}{2})R_y(\frac{\theta}{2})} & \targ & \gate{R_y(-\frac{\theta}{2})} & \targ & \gate{R_z(-\frac{\pi}{2})} & \qw
}$$

(2) $U=R_y(\theta)$:

$$\Qcircuit @C=1em @R=.7em {
    & \qw & \ctrl{1} & \qw & \ctrl{1} & \qw \\
    & \gate{R_y(\frac{\theta}{2})} & \targ & \gate{R_y(-\frac{\theta}{2})} & \targ & \qw
}$$

\ex pass.

\ex (1) $$\Qcircuit @C=1em @R=.7em {
    & \ctrl{1} & \ctrl{1} & \ctrl{1} & \qw \\
    & \ctrl{1} & \targ & \ctrl{1} & \qw \\
    & \targ & \ctrl{-1} & \targ & \qw
}$$

(2) $$\Qcircuit @C=1em @R=.7em {
    & \qw & \ctrl{1} & \qw & \qw \\
    & \ctrl{1} & \targ & \ctrl{1} & \qw \\
    & \targ & \ctrl{-1} & \targ & \qw
}$$

(3) Let $V=\frac{1}{2}\begin{pmatrix}
    1-i & 1+i \\
    1+i & 1-i
\end{pmatrix}$, then

$$
\Qcircuit @C=1em @R=.7em {
    & \qw & \ctrl{1} & \qw & \qw \\
    & \ctrl{1} & \targ & \ctrl{1} & \qw \\
    & \targ & \ctrl{-1} & \targ & \qw
}
=
\Qcircuit @C=1em @R=.7em {
    & \qw & \ctrl{1} & \qw & \qw \\
    & \targ & \ctrl{1} & \targ & \qw \\
    & \ctrl{-1} & \targ & \ctrl{-1} & \qw
}
=
\Qcircuit @C=1em @R=.7em {
    & \qw & \qw & \ctrl{1} & \qw & \ctrl{1} & \ctrl{2} & \qw & \qw \\
    & \targ & \ctrl{1} & \targ & \ctrl{1} & \targ & \qw & \targ & \qw \\
    & \ctrl{-1} & \gate{V} & \qw & \gate{V^\dagger} & \qw & \gate{V} & \ctrl{-1} & \qw
}
$$

(4) In the figure above, observing that the 6th gate is commutative with the 3rd, 4th, 5th gate, and that the last two CNOT gates are commutative, we can reformat it as   

$$\Qcircuit @C=1em @R=.7em {
    & \qw & \qw & \ctrl{2} & \ctrl{1} & \qw & \qw & \ctrl{1} & \qw \\
    & \targ & \ctrl{1} & \qw & \targ & \ctrl{1} & \targ & \targ & \qw \\
    & \ctrl{-1} & \gate{V} & \gate{V} & \qw & \gate{V^\dagger} & \ctrl{-1} & \qw & \qw
    \gategroup{2}{2}{3}{3}{.7em}{--}
    \gategroup{2}{6}{3}{7}{.7em}{--}
}$$

The gates in either dashed frame can be replaced by a single two-qubit gate, hence we obtain a 5 two-qubit gates construction.

Reference: \url{https://journals.aps.org/pra/pdf/10.1103/PhysRevA.53.2855}

\ex Fill all blanks with 4, then we can directly verify the proposition, with $\theta(c_1,c_2,t)=(-1)^{\neg c_1\wedge c_2}$.

\ex $$\Qcircuit @C=1em @R=.7em {
    & \ctrl{1} & \qw & \ctrl{2} & \targ & \ctrl{1} & \targ & \targ & \qw \\
    & \ctrl{1} & \ctrl{1} & \qw & \qw & \targ & \qw & \ctrl{-1} & \qw \\
    & \targ & \targ & \targ & \ctrl{-2} & \ctrl{-1} & \ctrl{-2} & \ctrl{-2} & \qw \\
    & & \mbox{flip z} & & & \mbox{flip y} & & \mbox{flip x} & {}
    \gategroup{1}{2}{3}{4}{.7em}{--}
    \gategroup{1}{5}{3}{7}{.7em}{--}
    \gategroup{1}{8}{3}{8}{.7em}{--}
}$$

\ex $$\Qcircuit @C=1em @R=.7em {
    & \qw &\ctrl{1} & \qw & \ctrl{1} & \ctrl{1} & \qw \\
    & \qw &\ctrl{1} & \qw & \ctrl{1} & \ctrl{1} & \qw \\
    & \qw &\ctrl{1} & \qw & \ctrl{1} & \ctrl{1} & \qw \\
    & \qw &\ctrl{1} & \qw & \ctrl{1} & \ctrl{2} & \qw \\
    & \ctrl{1} & \targ & \ctrl{1} & \targ & \qw & \qw \\
    & \gate{V} & \qw & \gate{V^\dagger} & \qw & \gate{V} & \qw
}$$

\ex I have found an answer that uses only $O(n)$ gates. \urlref{https://cs.stackexchange.com/questions/40933}

\ex Let $c_n$ be the number of gates needed. From the figure in (Ex. 4.28) we know 
$$\underbrace{c_n}_{C^n(U)} = \underbrace{c_{n-1}}_{C^{n-1}(V)} + 2\underbrace{O(n)}_{C^{n-1}(X)} + 2\underbrace{O(1)}_{C(V)\& C(V^\dagger)} = c_{n-1} + O(n).$$

By induction we have
$$c_n = O(n^2).$$

\ex pass.
