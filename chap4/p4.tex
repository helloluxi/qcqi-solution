\section*{Chapter Problems}
\addcontentsline{toc}{section}{Chapter Problems}

\prob Let $O(f)$ be the oracle implementing $(x,y)\rightarrow(x,y\oplus f(x))$, and controlled $U^k$-gate denote the gate group that for each $k\in\{1,2,\cdots,n\}$ the $k$-th control qubits controls a $diag(1,\exp(-2\pi i\frac{2^k}{2^n}))$-transform for the target qubit.
The total gate cost is $2T+n$.
$$\Qcircuit @C=1em @R=.7em {
    & \qw {/}^m & \multigate{1}{O(f)} & \qw & \multigate{1}{O(f)^\dagger} & \qw \\
    & \qw {/}^n & \ghost{O(f)} & \ctrl{1} & \ghost{O(f)^\dagger} & \qw \\
    & \qw & \qw & \gate{U^k} & \qw & \qw
}$$

\prob Let $k = \lceil \log n \rceil$.
First we add $2^k-n$ ancilla $\ket{0}$-state qubits to the input.
Then we can pack them into $2^{k-1}$ pairs, add an ancilla inited qubit and apply a Toffoli gate for each pair.
After $k$ iterations only one ancilla qubit is left, which is then used for controlling the target qubit.
Finally we uncompute the iterations.
The total depth is $2k+1$, which is $O(\log n)$.

\prob (1) pass.

(2) The freedom degree of $H$ is
$$4^n=\underbrace{2}_{complex}\times\underbrace{2^n\times(2^n-1)}_{non-diagonal}+\underbrace{2^n}_{real\& diagonal},$$
which is equal to the number of possibilities of $g$ is $4^n$.

(3) Analogous to Ex.4.51.

(4) Directly follows Eq.(4.103).

(5) $U=\exp(-iH)=\exp(-iH\Delta)^k=\left[\prod_g\exp(-ih_gg\Delta)\right]^k+O(4^n\Delta^2)^k=\left[\prod_g\exp(-ih_gg\Delta)\right]^k+O(4^n\Delta)$.

(6) $\varepsilon=\frac{4^n}{k}$ $\Rightarrow$ $k=4^n\varepsilon$ $\Rightarrow$ There are $4^n$ terms of $\exp(-ih_gg\Delta)$, each takes $O(n)$ operations, hence the total operations are $O(n16^n/\varepsilon)$.

\prob \todo

\prob \todo

\prob \todo
