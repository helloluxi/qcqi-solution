\section*{Chapter Problems}
\addcontentsline{toc}{section}{Chapter Problems}

\prob First apply the QFT circuit, but replace $R_k=\operatorname{diag}(1, e^{2\pi i/2^k})$ with $R_p=\operatorname{diag}(1, e^{2\pi i/p})$.

Then construct a comparing gate to calculate the Boolean function $f(x) = x < p$ in the way $\ket{x}\ket{0}\rightarrow\ket{x}\ket{f(x)}$, and apply it to the result qubits in the first step.

Finally measure the ancilla qubit, we obtain the desired state on condition that the measurement result is one.

\prob \todo

\prob $\ket{0}\ket{u} \rightarrow \frac{1+e^{2\pi i\varphi}}{2}\ket{0}\ket{u}+\frac{1-e^{2\pi i\varphi}}{2}\ket{1}\ket{u}$, hence the probability that the first qubit measurement returns 0 is $p=\left|\frac{1+e^{2\pi i\varphi}}{2}\right|^2=\cos^2(\pi\varphi)$. \todo

\prob \todo

\prob \todo

\prob Between QFT and iQFT, insert a $\begin{pmatrix}
    1 & 0 \\
    0 & e^{2\pi iy2^m/2^n}
\end{pmatrix}$-phase shift gate for the $m$-th qubit, it is easy to verify that the group of phase shifts transform $\ket{k}$ into $e^{2\pi iky/2^n}\ket{k}$.
Then the total circuit goes like,
$$
\ket{x} 
\overset{QFT}{\longrightarrow} 
\frac{1}{2^{n/2}}\sum_{k=0}^{2^n-1}e^{\frac{2\pi ikx}{2^n}}\ket{k} 
\overset{shift}{\longrightarrow} 
\frac{1}{2^{n/2}}\sum_{k=0}^{2^n-1}e^{\frac{2\pi ik(x+y)}{2^n}}\ket{k} 
\overset{iQFT}{\longrightarrow} 
\ket{x+y\operatorname{ mod }^n}.
$$

\redstar It seems that any values of $y$ can be added easily in this way?
