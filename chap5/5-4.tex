\section{General applications of the quantum Fourier transform}

\ex $$\begin{aligned}
    \hat{f}(l) = & \frac{1}{\sqrt{N}}\sum_{k\in\{0,r,\cdots,N-r\}}e^{-2\pi ilk/N}\left[\sum_{x=0}^{r-1}e^{-2\pi ilx/N}f(x)\right]
    \\ = & \begin{cases}
    \frac{\sqrt{N}}{r}\sum_{x=0}^{r-1}e^{-2\pi ilx/N}f(x),&\text{$l$ is an integer multiple of $N/r$}\\
    0,&\text{otherwise}
\end{cases}.
\end{aligned}$$ 

\redstar Consider $l=kN/r(k\in\mathbb{Z})$ we have
$$\hat{f}(k\frac{N}{r}) = \sqrt{\frac{N}{r}}\cdot\frac{1}{\sqrt{r}}\sum_{x=0}^{r-1}e^{-2\pi ikx/r}f(x).$$

\ex (1) $$\begin{aligned}
    U_y\ket{\hat{f}(l)}
    = & U_y\cdot\frac{1}{\sqrt{r}}\sum_{x=0}^{r-1}e^{-2\pi ilx/r}\ket{f(x)}
    = \frac{1}{\sqrt{r}}\sum_{x=0}^{r-1}e^{-2\pi ilx/r}\ket{f(x+y)}
    \\ = & \frac{1}{\sqrt{r}}e^{2\pi ily/r}\sum_{x=0}^{r-1}e^{-2\pi il(x+y)/r}\ket{f(x+y)}
    = e^{2\pi ily/r}\ket{\hat{f}(l)}
\end{aligned}$$

(2) $$\begin{aligned}
    U_y\ket{f(x_0)}
    = & \cdot\frac{1}{\sqrt{r}}\sum_{x=0}^{r-1}e^{2\pi ilx_0/r}U_y\ket{\hat{f}(l)}
    \\ = & \frac{1}{\sqrt{r}}e^{2\pi ily/N}\sum_{l=0}^{r-1}e^{2\pi ilx_0/N}\ket{\hat{f}(l)}
    \\ \overset{iQFT}{\longrightarrow} & \frac{1}{\sqrt{r}}e^{2\pi ily/N}\sum_{l=0}^{r-1}\ket{\tilde{l/r}}\ket{\hat{f}(l)}
\end{aligned}$$

\ex $$\begin{aligned}
    \ket{\hat{f}(l)}
    = & \frac{1}{r}\sum_{x_1=0}^{r-1}\sum_{x_2=0}^{r-1}e^{-2\pi i(l_1x_1+l_2x_2)/r}\ket{f(x_1,x_2)}
    \\ = & \frac{1}{r}\sum_{x_1=0}^{r-1}\sum_{x_2=0}^{r-1}e^{-2\pi i(l_1x_1+l_2x_2)/r}\ket{f(0,x_2+sx_1)}
    \\ = & \frac{1}{r}\sum_{x_1=0}^{r-1}\sum_{j=sx_1}^{r-1+sx_1}e^{-2\pi i(l_1x_1+l_2(j-sx_1))/r}\ket{f(0,j)}
    \\ = & \frac{1}{r}\sum_{x_1=0}^{r-1}e^{-2\pi isx_1(l_1/s-l_2)/r}\sum_{j=0}^{r-1}e^{-2\pi il_2j/r}\ket{f(0,j)}
    \\ = & \sum_{j=0}^{r-1}e^{-2\pi il_2j/r}\ket{f(0,j)}
\end{aligned}$$

\ex $$\begin{aligned}
    LHS
    = & \frac{1}{r} \sum_{l_1=0}^{r-1} \sum_{l_2=0}^{r-1} \sum_{j=0}^{r-1} e^{2\pi i(l_1x_1+l_2x_2-l_2j)/r} \ket{f(0,j)}
    \\ = & \frac{1}{r} \sum_{l_2=0}^{r-1} \sum_{j=0}^{r-1} e^{2\pi il_2(sx_1+x_2-j)/r} \ket{f(0,j)}
    \\ = & \ket{f(0,x_2+sx_1)} = \ket{f(x_1,x_2)}
\end{aligned}$$

\ex \todo

\ex \todo

\ex {\color{red}Can't understand at all.}

\ex \todo

\ex \todo

\ex \todo
