\section{Examples of quantum noise and quantum operations}

\ex If not, i.e., $\det(O)=-1$, then we replace $O,S$ by $-O,-S$.

\ex pass.

\ex pass.

\ex Let $\rho=(I+\vec{r}\cdot\vec{\sigma})/2$.
Then $$\rho=\frac{1+r_x}{2}\dyadic{+}{+}+\frac{1-r_x}{2}\dyadic{-}{-}+(r_z-ir_y)\dyadic{+}{-}+(r_z+ir_y)\dyadic{-}{+}.$$

And then,
$$\mathcal{E}(\rho) = \frac{1+r_x}{2}\dyadic{+}{+}+\frac{1-r_x}{2}\dyadic{-}{-} = \frac{I+(\hat{x}\cdot\vec{r})\hat{x}\cdot\vec{\sigma}}{2}.$$

So $\mathcal{E}$ acts on the Bloch sphere by projecting it into the $x$-axis.

\ex Any non-trace-preserving quantum operation whose output dosen't have the form $(I+\vec{r}\cdot\vec{\sigma})/2$.

\ex The verification of Eq.~(8.105) is omitted.
For any density operator $\rho=(I+\vec{r}\cdot\vec{\sigma})/2$, since $\mathcal{E}$ is a linear operator, we have
$$\mathcal{E}(\rho) = \frac{\mathcal{E}(I) + r_x\mathcal{E}(X) + r_y\mathcal{E}(Y) + r_z\mathcal{E}(Z)}{2} = \frac{I}{2}.$$

\ex Let $\rho=(I+\vec{r}\cdot\vec{\sigma})/2$, the conclusion is obvious when $\vec{r}=0$.
When $\vec{r}\neq 0$, we have
$$\rho=\frac{1-r}{2}I+r\frac{I+\hat{r}\cdot\vec{\sigma}}{2}.$$

Notice that $[(I+\hat{r}\cdot\vec{\sigma})/2]^k$ is constant for any $k\in\mathbb{Z}_+$, we have
$$\tr\left(\frac{I+\hat{r}\cdot\vec{\sigma}}{2}\right)^k=\begin{cases}2,k=0\\1,k>0\end{cases}.$$

Therefore,
$$\begin{aligned}
    \tr(\rho^k) = & \tr\left[ \sum_{j=0}^k C_k^j \left(\frac{1-r}{2}\right)^{k-j} r^j \left(\frac{I+\hat{r}\cdot\vec{\sigma}}{2}\right)^j \right]
    \\ = & 2\left(\frac{1-r}{2}\right)^k + \sum_{j=1}^k C_k^j \left(\frac{1-r}{2}\right)^{k-j} r^j
    \\ = & \left(\frac{1-r}{2}\right)^k + \left(\frac{1+r}{2}\right)^k.
\end{aligned}$$

Define 
$$
    f(r) := \left(\frac{1-r}{2}\right)^k + \left(\frac{1+r}{2}\right)^k
    = \frac{1}{2^k}\sum_{j\text{ even}}C_k^jr^j,
$$
then we can see $f(r)$ is increasing about $r$.

Meanwhile 
$$\mathcal{E}(\rho)=\frac{1-(1-p)r}{2} I + (1-p)r \frac{I+\hat{r}\cdot\vec{\sigma}}{2}.$$

So
$$\tr(\mathcal{E}(\rho)^k) = f((1-p)r) \le f(r) = \tr(\rho^k).$$

\ex \todo

\ex The overall unitary matrix of the circuit is
$$
    \begin{pmatrix}
        1 & & & \\
        & & & 1 \\
        & & 1 & \\
        & 1 & & 
    \end{pmatrix}
    \begin{pmatrix}
        1 & & & \\
        & 1 & & \\
        & & \cos\frac{\theta}{2} & -\sin\frac{\theta}{2} \\
        & & \sin\frac{\theta}{2} & \cos\frac{\theta}{2}
    \end{pmatrix}
    =
    \begin{pmatrix}
        1 & & & \\
        & & \sin\frac{\theta}{2} & \cos\frac{\theta}{2} \\
        & & \cos\frac{\theta}{2} & -\sin\frac{\theta}{2} \\
        & 1 & & 
    \end{pmatrix}.
$$

Hence
$$E_0 = \bra{0_2}U\ket{0_2} = \begin{pmatrix}
    U_{11} & U_{13} \\
    U_{31} & U_{33}
\end{pmatrix} = \begin{pmatrix}
    1 & 0 \\
    0 & \cos\frac{\theta}{2}
\end{pmatrix},$$

$$E_1 = \bra{1_2}U\ket{0_2} = \begin{pmatrix}
    U_{21} & U_{23} \\
    U_{41} & U_{43}
\end{pmatrix} = \begin{pmatrix}
    0 & \sin\frac{\theta}{2} \\
    0 & 0
\end{pmatrix}.$$

\ex (1) \urlref{https://quantumcomputing.stackexchange.com/questions/6828}

(2) $$\begin{aligned}
    \sum_kE_k^\dagger E_k = & \sum_{n\ge k}\sum_k C_n^k (1-\gamma)^{n-k}\gamma^k \dyadic{n}{n}
    \\ = & \sum_n\sum_{m=0}^n C_n^m (1-\gamma)^m\gamma^{n-m} \dyadic{n}{n}(\text{Let }m=n-k)
    \\ = & \sum_n\dyadic{n}{n}
    \\ = & I.
\end{aligned}$$

\ex $$\mathcal{E}_{AD}(\rho)=\begin{pmatrix}
    a+\gamma c & \sqrt{1-\gamma} b \\
    \sqrt{1-\gamma} b^* & (1-\gamma) c
\end{pmatrix}.$$

\ex $$\begin{aligned}
    & \mathcal{E}_{AD}\otimes\mathcal{E}_{AD}(\dyadic{\psi}{\psi})
    \\ = & \mathcal{E}_{AD}\otimes\mathcal{E}_{AD}(|a|^2\dyadic{01}{01}+ab^*\dyadic{01}{10}+a^*b\dyadic{01}{01}+|b|^2\dyadic{10}{10})
    \\ = & |a|^2\mathcal{E}_{AD}(\dyadic{0}{0})\otimes\mathcal{E}_{AD}(\dyadic{1}{1}) + |b|^2\mathcal{E}_{AD}(\dyadic{1}{1})\otimes\mathcal{E}_{AD}(\dyadic{0}{0})
    \\ & + ab^*\mathcal{E}_{AD}(\dyadic{0}{1})\otimes\mathcal{E}_{AD}(\dyadic{1}{0}) + a^*b\mathcal{E}_{AD}(\dyadic{1}{0})\otimes\mathcal{E}_{AD}(\dyadic{0}{1})
    \\ = & |a|^2 \begin{pmatrix} 1 & 0 \\ 0 & 0 \end{pmatrix} \otimes \begin{pmatrix} \gamma & 0 \\ 0 & 1-\gamma \end{pmatrix} + |b|^2 \begin{pmatrix} \gamma & 0 \\ 0 & 1-\gamma \end{pmatrix} \otimes \begin{pmatrix} 1 & 0 \\ 0 & 0 \end{pmatrix}
    \\ & + ab^*(1-\gamma) \begin{pmatrix} 0 & 1 \\ 0 & 0 \end{pmatrix} \otimes \begin{pmatrix} 0 & 0 \\ 1& 0 \end{pmatrix} + a^*b(1-\gamma) \begin{pmatrix} 0 & 0 \\ 1 & 0 \end{pmatrix} \otimes \begin{pmatrix} 0 & 1 \\ 0 & 0 \end{pmatrix}
    \\ = & \begin{pmatrix}
        \gamma & 0 & 0 & 0 \\
        0 & |a|^2(1-\gamma) & ab^*(1-\gamma) & 0 \\
        0 & a^*b(1-\gamma) & |a|^2(1-\gamma) & 0 \\
        0 & 0 & 0 & 0
    \end{pmatrix}
    \\ = & (1-\gamma)\dyadic{\psi}{\psi} + \gamma[\text{tranform into }\dyadic{00}{00}].
\end{aligned}$$

\ex \todo

\ex $$T=k_B\frac{E_1-E_0}{\ln\frac{p}{1-p}}.$$

\ex The overall unitary matrix of the circuit is
$$\begin{pmatrix}
    1 & & & \\
    & 1 & & \\
    & & \cos\frac{\theta}{2} & -\sin\frac{\theta}{2} \\
    & & \sin\frac{\theta}{2} & \cos\frac{\theta}{2}
\end{pmatrix}.$$

Hence
$$E_0 = \bra{0_2}U\ket{0_2} = \begin{pmatrix}
    1 & 0 \\
    0 & \cos\frac{\theta}{2}
\end{pmatrix},$$

$$E_1 = \bra{1_2}U\ket{0_2} = \begin{pmatrix}
    0 & 0 \\
    0 & \sin\frac{\theta}{2}
\end{pmatrix}.$$

\ex By $\tilde{E}_i\sum_ju_{ij}E_j$ we have
$$\begin{cases}
    u_{00} & = \sqrt{\frac{1+\sqrt{1-\lambda}}{2}}, \\
    \sqrt{1-\lambda} u_{00} + \sqrt{\lambda} u_{01} & = \sqrt{\frac{1+\sqrt{1-\lambda}}{2}}, \\
    u_{10} & = \sqrt{\frac{1-\sqrt{1-\lambda}}{2}}, \\
    \sqrt{1-\lambda} u_{10} + \sqrt{\lambda} u_{11} & = -\sqrt{\frac{1-\sqrt{1-\lambda}}{2}}.
\end{cases}$$

The solution is
$$
u=\begin{pmatrix}
    \sqrt{\frac{1+\sqrt{1-\lambda}}{2}} & \sqrt{\frac{1-\sqrt{1-\lambda}}{2}} \\
    \sqrt{\frac{1-\sqrt{1-\lambda}}{2}} & -\sqrt{\frac{1+\sqrt{1-\lambda}}{2}}
\end{pmatrix}.
$$

\ex \todo

\ex pass.

\ex \redstar From Ex. 8.22 we see the off-diagonal elements decay square-rootly as diagonal elements, hence $T_2=2T_1$?

Phase damping only accelarates off-diagonal elements decaying, which makes $T_2\le 2T_1$?

\ex \todo